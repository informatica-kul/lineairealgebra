\documentclass[lineaire_algebra_oplossingen.tex]{subfiles}
\begin{document}

\section{Zelfreflectie 4}

\subsection{Oefening 1}
\[
L: \mathbb{R}^n \rightarrow \mathbb{R}: (x_1,x_2,...,x_n) \mapsto \sum_{i=1}^n\lambda_ix_i
\]

\subsection{Oefening 2}
Ja!
\begin{proof}
De verzameling van alle niet-lineaire afbeeldingen is een deelruimte van alle afbeeldingen als de lineaire combinatie intern is. De lineaire combinatie van niet-lineaire afbeeldingen is uiteraard een niet-lineaire afbeelding.
\end{proof}

\subsection{Oefening 3}
\subsubsection*{a)}
Ja! Dit volgt meteen uit de invloed van basiskeuze. Mits een verschillend gekozen basis kunnen verschillende lineaire afbeeldingen dezelfde matrix beschrijven.

\subsubsection*{b)}
Ja! Idem.

\subsection{Oefening 4}
Zij $(\mathbb{R},V,+)$ en $(\mathbb{R},W,+)$ eindig dimensionale vectorruimten.
\subsubsection*{Te Bewijzen}
\begin{center}
$V$ en $W$ zijn isomorf.
\end{center}
\[\Leftrightarrow\]
\[dim(V)=dim(W)\]
\subsubsection*{Bewijs}
\begin{proof}
Bewijs van een equivalentie.
\begin{itemize}
\item $\Rightarrow$\\
Als $V$ en $W$ isomorf zijn bestaat er een lineaire bijectieve afbeelding $L:V\rightarrow W$. Omdat $L$ bijectief is moeten $dim(V)$ en $dim(W)$ gelijk zijn.
\item $\Leftarrow$\\
Omdat $dim(V)$ gelijk is aan $dim(W)$ bestaan er lineaire bijectieve co\"ordinaatafbeeldingen $a$ en $b$ van respectievelijk $V$, $W$ naar $co_\beta$ voor elke basis van $V$ respectievelijk $W$\footnote{Zie Stelling 3.45 p 111}. Deze afbeeldingen zijn inverteerbaar omdat ze bijectief zijn. De samenstelling $b^{-1}\circ a$ is opnieuw bijectief en lineair\footnote{Zie Stelling 4.13 p 142}. Omdat deze nieuwe afbeelding bijectief en lineair is is het een isomorfisme, dus zijn $V$ en $W$ isomorf.
\end{itemize}
\end{proof}

\subsection{Oefening 5}
Zie \ref{matrix_van_lineaire_afbeeldingen_tov_gegeven_basissen} voor meer uitleg.
$L_{\alpha}^\beta$ ziet er als volgt uit.
\[
L_{\alpha}^\beta = Id_{\epsilon_2}^\beta \cdot L_{\epsilon_1}^{\epsilon_2} \cdot Id_{\alpha}^{\epsilon_1}
\]

\subsection{Oefening 6}
$L_{\alpha}^\beta$ en $Id_{\alpha}^\beta$ zijn gelijk omdat $L$ een lineaire transformatie is.

\subsection{Oefening 7}
Zie \ref{4.39}.

\subsection{Oefening 8}
Ja, want $L$ behoudt de lineaire combinatie.

\subsection{Oefening 9}
Zij $V_1$ en $V_2$ vectorruimten en $L$ een lineaire afbeelding van $V_1$ naar $V_2$. Zijn $U_2$ een deelruimte van $V_2$

\subsubsection*{Te Bewijzen}
$U_1$ is een deelruimte van $V_1$.
\[
U_1 = \{x \in V_1\ |\  L(x) \in U_2 \}
\]

\subsubsection*{Bewijs}
\begin{proof}
Kies willekeurige $u_1, u_2 \in U_1$ en $\lambda_1,\lambda_2 \in \mathbb{R}$.
\[
\lambda_1u_1 + \lambda_2u_2 \in U_1
\]
Bovenstaande bewering geldt omdat ze equivalent is aan de volgende, en $U_2$ is een deelruimte van $V_2$.
\[
L(\lambda_1u_1 + \lambda_2u_2)
=\lambda_1L(u_1) + \lambda_2L(u_2)
 \in U_2
\]
\end{proof}

\subsection{Oefening 10}
$Hom_{\mathbb{R}}(V,W)$ is isomorf met $\mathbb{R}^{m\times n}$.

\subsection{Oefening 11}
Waar, let op de indices.

\subsection{Oefening 12}
Overal waar je zegt ``...\emph{de matrix van basisverandering}...'' in plaats van ``... een matrix van basisverandering...''.

\subsection{Oefening 13}
De lineaire afbeeldingen die overeenkomen met de standaardbasis van $\mathbb{R}^{m\times n}$ kunnen als basis dienen voor $Hom_\mathbb{R}(V,W)$.

\subsection{Oefening 14}
Het bewijs is volledig analoog. (Zie \ref{4.35}.)

\subsection{Oefening 15}
Zie \ref{4.34}.

\subsection{Oefening 16}
Zij $x_p$ een particuliere oplossing van het stelsel.
De oplossingsverzameling van het stelsel is dan de oorsprong,  een rechte door $x_p$ of een vlak door $x_p$.


\end{document}