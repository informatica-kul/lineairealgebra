\documentclass[lineaire_algebra_oplossingen.tex]{subfiles}
\begin{document}

%TODO Hermaak dit als je je geroepen voelt. Er staan nog fouten in (zie de opmerkingen eronder). Voel je vrij om het te verbeteren.

\chapter{Diagrammen}
\begin{figure}[H]
\begin{mdframed}
\begin{tikzpicture}
\node (bl)  at (0,5) {};  
\node (tl)  at (0,12) {};
\node (tr)  at (9,12) {};
\node (br)  at (9,5) {};

\node (bl2)  at (-2,9) {};  
\node (tl2)  at (-2,11.5) {};
\node (tr2)  at (9,11.5) {};
\node (br2)  at (9,9) {};

\node (bl3)  at (-1.5,9) {};  
\node (tl3)  at (-1.5,11) {};
\node (tr3)  at (7,11) {};
\node (br3)  at (7,9) {};

\node (bl4)  at (-1,9) {};  
\node (tl4)  at (-1,10) {};
\node (tr4)  at (7,10) {};
\node (br4)  at (7,9) {};

\node (x1)  at (8,8) {};  
\node (x2)  at (8,7) {}; 
\node (x3)  at (6,7) {}; 

\node (bl5)  at (-3,9) {};  
\node (tl5)  at (-3,12) {};
\node (tr5)  at (9,12) {};
\node (br5)  at (9,9) {};

\node (x4)  at (9,7) {}; 

\node (bl6)  at (7.25,7) {};  
\node (tl6)  at (7.25,11.75) {};
\node (tr6)  at (7.75,11.75) {};
\node (br6)  at (8.25,7) {};
\node (wtf)  at (7.75,7) {};

\begin{scope}[fill opacity=0.5]
	\filldraw[fill=green!15] ($(bl5) + (-1,-0)$) 
    	to ($(tl5) + (-1,0)$)
        to[out=90,in=180] ($(tl5) + (0,1)$) 
        to ($(tr5) + (0,1)$)
        to[out=0,in=90] ($(tr5) + (1,0)$)
        to ($(br5) + (1,0)$)
        to[out=270,in=0] ($(br5) + (0,-1)$)
        to ($(bl5) + (0,-1)$)
        to[out=180,in=270] ($(bl5) + (-1,0) $);
        
    \filldraw[fill=yellow!25] ($(bl) + (-1,-0)$) 
    	to ($(tl) + (-1,0)$)
        to[out=90,in=180] ($(tl) + (0,1)$) 
        to ($(tr) + (0,1)$)
        to[out=0,in=90] ($(tr) + (1,0)$)
        to ($(br) + (1,0)$)
        to[out=270,in=0] ($(br) + (0,-1)$)
        to ($(bl) + (0,-1)$)
        to[out=180,in=270] ($(bl) + (-1,0) $);
        
    \filldraw[fill=red!25] ($(bl2) + (-1,-0)$) 
    	to ($(tl2) + (-1,0)$)
        to[out=90,in=180] ($(tl2) + (0,1)$) 
        to ($(tr2) + (0,1)$)
        to[out=0,in=90] ($(tr2) + (1,0)$)
        to ($(br2) + (1,0)$)
        to[out=270,in=0] ($(br2) + (0,-1)$)
        to ($(bl2) + (0,-1)$)
        to[out=180,in=270] ($(bl2) + (-1,0) $);
        
    \filldraw[fill=black!15] ($(bl6) + (-1,-0)$) 
    	to ($(tl6) + (-1,0)$)
        to[out=90,in=180] ($(tl6) + (0,1)$) 
        to ($(tr6) + (0,1)$)
        to[out=0,in=90] ($(tr6) + (1,0)$)
        to ($(wtf) + (1,1)$)
        to ($(br6) + (1,0)$)
        to[out=270,in=0] ($(br6) + (0,-1)$)
        to ($(bl6) + (0,-1)$)
        to[out=180,in=270] ($(bl6) + (-1,0) $);
        
    \filldraw[fill=green!25] ($(bl3) + (-1,-0)$) 
    	to ($(tl3) + (-1,0)$)
        to[out=90,in=180] ($(tl3) + (0,1)$) 
        to ($(tr3) + (0,1)$)
        to[out=0,in=90] ($(tr3) + (1,0)$)
        to ($(br3) + (1,0)$)
        to[out=270,in=0] ($(br3) + (0,-1)$)
        to ($(bl3) + (0,-1)$)
        to[out=180,in=270] ($(bl3) + (-1,0) $);
            
    \filldraw[fill=blue!25] ($(bl4) + (-1,-0)$) 
    	to ($(tl4) + (-1,0)$)
        to[out=90,in=180] ($(tl4) + (0,1)$) 
        to ($(tr4) + (0,1)$)
        to[out=0,in=90] ($(tr4) + (1,0)$)
        to ($(br4) + (1,0)$)
        to[out=270,in=0] ($(br4) + (0,-1)$)
        to ($(bl4) + (0,-1)$)
        to[out=180,in=270] ($(bl4) + (-1,0) $);
        
    \filldraw[fill=green!25] ($(x1) + (-1,0)$) 
    	to ($(x1) + (1,0)$)
    	to ($(x1) + (0,1)$) 
    	to[out=270,in=0] ($(x1) + (-1,0)$); 
    	
    \filldraw[fill=purple!25] ($(x1) + (-0.8,0)$) 
    	to ($(x2)$)
    	to ($(x2) + (1,0)$)
    	to ($(x1) + (1,0)$)
    	to ($(x1) + (0,1)$) 
    	to[out=270,in=0] ($(x1) + (-1,0)$);
    	
    \filldraw[fill=purple!25] ($(x3) + (1.2,1)$)
    	to ($(x3) + (0,1)$)
    	to ($(x3)$)
    	to ($(x3) + (1,0)$)
    	to ($(x3) + (1.2,1)$);
    \filldraw[fill=blue!25] ($(x1) + (-0.8,0)$)
    	to ($(x4) + (0,1)$) 
    	to ($(x4)$)
    	to ($(x4) + (1,0)$)
    	to ($(x4) + (1,1)$)
    	to ($(x1) + (1,0)$)
    	to ($(x1) + (0,1)$) 
    	to[out=270,in=0] ($(x1) + (-1,0)$);    
\end{scope}

\node at (-4,4) {$\mathbb{R}^{m\times n}$};
\node at (0,5) {$R^{n\times n}$};      
\node at (9.2,12.0) {$TV$};   
\node at (7,11.5) {$EV$};   
\node at (7,10.5) {$RR$};    
\node at (8.2,8.4) {$Diag$}; 
\node at (8.2,7.4) {$Sym$}; 
\node at (6.5,7.4) {$SSym$}; 
\node at (-3.4,12.5) {$BoD$};
\node at (9.5,7.4) {$BeD$}; 
\node at (8.5,6.5) {$Inv$}; 
\node (v1) at (7.5,8.5) {$\mathbb{I}_n$}; 
\node (v2) at (7.2,8.4) {$O$}; 


\foreach \v in {1,2,...,2} {
   \fill (v\v)+(0,-0.4) circle (0.05);
}
\end{tikzpicture}
\end{mdframed}
\label{matrices}
\caption{Matrices}
\end{figure}
In bovenstaande figuur (\ref{matrices}) staat een Venn diagram van matrices en hun eigenschappen.
Merk op:
\begin{itemize}
\item De verzamelingen zijn niet op schaal getekend.
\item $Diag \subset BeD$ geldt, maar dit is niet heel duidelijk op de figuur omdat het moeilijk te tekenen is.
\item $Diag \subset Sym$ geldt ook, maar heeft hetzelfde probleem.
\item Dit diagram gaat ervan uit de nulrijen in een diagonaalmatrix steeds onderaan staan zodat het makkelijker te tekenen is.
\end{itemize}


\begin{figure}[H]
\centering
\[
\begin{array}{c | c c c c c c c c c c c}
 & \mathbb{R}^{m\times n} & \mathbb{R}^{n\times n} & TV & EV & RR & Sym & SSym & Diag & BoD & BeD & Inv\\
 \hline
\mathbb{R}^{m\times n} 	& =\\
\mathbb{R}^{n\times n} 	& \subset 	& =\\
TV 						& \subset 	& \cdot 	& =\\
EV 						& \subset 	& \cdot 	& \subset 	& =\\
RR 						& \subset 	& \cdot 	& \subset 	& \subset 			& =\\ 
Sym 					& \subset	& \subset	& Diag		& \{\mathbb{I}_n,O\}& \{\mathbb{I}_n,O\}& =\\
SSym 					& \subset	& \subset	& \{O\}		& \{O\}				& \{O\} 			& \{O\}	& =\\
Diag 					& \subset	& \cdot		& \subset	& \{\mathbb{I}_n,O\}& \{\mathbb{I}_n,O\}& \subset	& \{O\}	& =\\
BoD 					& \subset	& \cdot		& \supset	& \supset 			& \supset			& Diag				& \{O\}	& \supset	& =\\
BeD 					& \subset	& \cdot		& Diag		& \{\mathbb{I}_n,O\}& \{\mathbb{I}_n,O\}& Diag				& \{O\} & \supset	& Diag	& =\\
Inv						& \subset	& \subset	& \cdot		& \cdot 			& \cdot 			& \cdot	& \cdot 		& \cdot	& \cdot		& \cdot 	& =\\
\end{array}
\]
\[
\begin{array}{c | c c c c c c c c c c c}
 & \mathbb{R}^{m\times n} & \mathbb{R}^{n\times n} & TV & EV & RR & Sym & SSym & Diag & BoD & BeD & Inv\\
\hline
\mathbb{I}_n 			& \in & \in & \in & \in & \in & \in & \not\in 	& \in & \in & \in & \in \\
O 						& \in & \in & \in & \in & \in & \in & \in 		& \in & \in & \in & \not\in\\
\end{array}
\]
\label{tabel}
\caption{Tabel Ter Verduidelijking}
\end{figure}


\begin{figure}[H]
\centering
\begin{tabular}{| c | c |}
\hline
Afkorting & Betekenis\\
\hline
TV & Trapvorm\\
EV & Echelonvorm\\
RR & Rijgereduceerd\\
Sym & Symmetrisch\\
SSym & Scheef-Symmetrisch\\
Diag & Diagonaal\\
BoD & Bovendriehoeks\\
BeD & BenedenDriehoeks\\
Inv & Inverteerbaar\\
\hline
\end{tabular}
\label{legende}
\caption{Legende}
\end{figure}






\iffalse
\begin{tikzpicture}
    \node (v1) at (0,2) {};
    \node (v2) at (1.5,3) {};
    \node (v3) at (4,2.5) {};
    \node (v4) at (0,0) {};
    \node (v5) at (2,0.5) {};
    \node (v6) at (3.5,0) {};
    \node (v7) at (2.5,-1) {};

    \begin{scope}[fill opacity=0.8]
    \filldraw[fill=yellow!70] ($(v1)+(-0.9,0)$) 
        to[out=90,in=180] ($(v2) + (0,0.5)$) 
        to[out=0,in=90] ($(v3) + (1,0)$)
        to[out=270,in=0] ($(v2) + (1,-0.8)$)
        to[out=180,in=270] ($(v1)+(-0.9,0)$);
    \filldraw[fill=blue!70] ($(v4)+(-0.5,0.2)$)
        to[out=90,in=180] ($(v4)+(0,1)$)
        to[out=0,in=90] ($(v4)+(0.6,0.3)$)
        to[out=270,in=0] ($(v4)+(0,-0.6)$)
        to[out=180,in=270] ($(v4)+(-0.5,0.2)$);
    \filldraw[fill=green!80] ($(v5)+(-0.5,0)$)
        to[out=90,in=225] ($(v3)+(-0.5,-1)$)
        to[out=45,in=270] ($(v3)+(-0.7,0)$)
        to[out=90,in=180] ($(v3)+(0,0.5)$)
        to[out=0,in=90] ($(v3)+(0.7,0)$)
        to[out=270,in=90] ($(v3)+(-0.3,-1.8)$)
        to[out=270,in=90] ($(v6)+(0.5,-0.3)$)
        to[out=270,in=270] ($(v5)+(-0.5,0)$);
    \filldraw[fill=red!70] ($(v2)+(-0.5,-0.2)$) 
        to[out=90,in=180] ($(v2) + (0.2,0.4)$) 
        to[out=0,in=180] ($(v3) + (0,0.3)$)
        to[out=0,in=90] ($(v3) + (0.3,-0.1)$)
        to[out=270,in=0] ($(v3) + (0,-0.3)$)
        to[out=180,in=0] ($(v3) + (-1.3,0)$)
        to[out=180,in=270] ($(v2)+(-0.5,-0.2)$);
    \end{scope}


    \foreach \v in {1,2,...,7} {
        \fill (v\v) circle (0.1);
    }

    \fill (v1) circle (0.1) node [right] {$v_1$};
    \fill (v2) circle (0.1) node [below left] {$v_2$};
    \fill (v3) circle (0.1) node [left] {$v_3$};
    \fill (v4) circle (0.1) node [below] {$v_4$};
    \fill (v5) circle (0.1) node [below right] {$v_5$};
    \fill (v6) circle (0.1) node [below left] {$v_6$};
    \fill (v7) circle (0.1) node [below right] {$v_7$};

    \node at (0.2,2.8) {$e_1$};
    \node at (2.3,3) {$e_2$};
    \node at (3,0.8) {$e_3$};
    \node at (0.1,0.7) {$e_4$};
\end{tikzpicture}
\fi

\end{document}
