\documentclass[lineaire_algebra_oplossingen.tex]{subfiles}
\begin{document}

\section{Examen Augustus 2012}

\subsection{Vraag 1 (Theorie)}
Zie Stelling 3.49 p. 114 (\ref{3.49}).

\subsection{Vraag 2 (Theorie)}
Zie Stelling 5.18 p. 190 (\ref{5.18}).

\subsection{Vraag 3}
\[
A^n + a_{n-1}A^{n-1} + ... + a_1A + a_0\mathbb{I}_n = 0
\]
\subsection*{(a)}
\[
A^n + a_{n-1}A^{n-1} + ... + a_1A + a_0\mathbb{I}_n = 0
\]
\begin{proof}
Bewijs van een equivalentie.
\[
A(A^{n-1} + a_{n-1}A^{n-2} + ... + a_1\mathbb{I}_n) = -a_0\mathbb{I}_n
\]
\begin{itemize}
\item $\Rightarrow$
Stel $A$ is inverteerbaar en $a_0$ is \emph{wel} nul.
\[
A^{n-1} + a_{n-1}A^{n-2} + ... + a_2A + a_1\mathbb{I}_n = 0
\]
Maar $n$ is het kleinste getal waarvoor dit soort gelijkheid geldt, contradictie.

\item $\Leftarrow$
Stel $a_0 \neq 0$ en $A$ \emph{niet} inverteerbaar. $\det(A) = 0$
\[
\det(A(A^{n-1} + a_{n-1}A^{n-2} + ... + a_1\mathbb{I}_n)) = \det(-a_0\mathbb{I}_n)
\]
\[
\det(A)\det((A^{n-1} + a_{n-1}A^{n-2} + ... + a_1\mathbb{I}_n)) = \det(-a_0\mathbb{I}_n)
\]
\[
0 = -a_0
\]
Contradictie.
\end{itemize}
\end{proof}

\subsubsection*{(b)}
\[
(A^n + a_{n-1}A^{n-1} + ... + a_1A + a_0\mathbb{I}_n)v = 0
\]
\[
A^nv + a_{n-1}A^{n-1}v + ... + a_1Av + a_0\mathbb{I}_nv = 0
\]
\[
\lambda^nv + a_{n-1}\lambda^{n-1}v + ... + a_1\lambda v + a_0v = 0
\]
\[
(\lambda^n + a_{n-1}\lambda^{n-1} + ... + a_1\lambda + a_0)v = 0
\]
\[
\lambda^n + a_{n-1}\lambda^{n-1} + ... + a_1\lambda + a_0 = 0
\]
$\lambda$, een eigenwaarde van $A$, voldoet aan de vergelijking.

\subsection{Vraag 4}
\subsubsection*{(a)}
\begin{proof}
De kern van $L$ is gelijk aan de nulruimte van $A$. Bovendien is de nulruimte van $A$ het orthogonaal complement van de rijruimte van $A$, of de kolomruimte van $A^T$. De kolomruimte van $A^T$ is precies het beeld van $L^T$.
\[
Ker(L) = Im(L^T)^\bot
\]
\end{proof}


\subsubsection*{(b)}
\[
Im(L^T) \oplus Im(L^T)^\bot = \mathbb{R}^n
\]
\[
dim(Im(L^T)) + dim(Im(L^T)^\bot) = \mathbb{R}^n
\]
\[
dim(Im(L^T)) + dim(Ker(L)) = \mathbb{R}^n
\]
\[
dim(Im(L^T)) = \mathbb{R}^n - dim(Ker(L))
\]
\[
dim(Im(L^T)) = dim(Im(L))
\]

\subsection{Vraag 5}
We rijreduceren eerst volgende matrix om de dimensie van $vct(D)$ te vinden.
\[
\begin{pmatrix}
1 & 0 & 2012 & 7\\
0 & a & 0 & 0\\
a & 0 & a & 3\\
0 & a^2 & 2011 & 1\\
2 & a & a-9 & 5
\end{pmatrix}
\rightarrow
\begin{pmatrix}
1 & 0 & 2012 & 7\\
0 & a & 0 & 0\\
a & 0 & a & 3\\
0 & 0 & 2011 & 1\\
2 & 0 & a-9 & 5
\end{pmatrix}
\rightarrow
\begin{pmatrix}
1 & 0 & 2012 & 7\\
0 & a & 0 & 0\\
0 & 0 & -2011a & 3\\
0 & 0 & 2011 & 1\\
0 & 0 & a-4033 & -9
\end{pmatrix}
\]
Lelijk he? Met een beetje inzicht kan het simpeler. We onderscheiden $a=0$ en $a\neq 0$
\subsubsection*{Geval 1: $a=0$}
\[
\begin{pmatrix}
1 & 0 & 2012 & 7\\
0 & 0 & 0 & 0\\
0 & 0 & 0 & 3\\
0 & 0 & 2011 & 1\\
2 & 0 & -9 & 5
\end{pmatrix}
\]
Neem de eerste, derde en vierde vector als basis.

\subsubsection*{Geval 2: $a\neq0$}
\[
\begin{pmatrix}
1 & 0 & 2012 & 7\\
0 & 1 & 0 & 0\\
a & 0 & a & 3\\
0 & 0 & 2011 & 1\\
2 & 0 & a-9 & 5
\end{pmatrix}
\]
Neem alle vier de vectoren als basis.

\subsection{Vraag 6}
\subsubsection*{(a)}
\begin{enumerate}[i.]
\item
Fout. Tegenvoorbeeld:
\[
\left\{
\begin{array}{c}
R_1\\R_2\\R_3
\end{array}
\right.
=
\left\{
\begin{array}{c}
x=0\\
y=0\\
z=0
\end{array}
\right.
\]
De oplossingsverzameling hiervan is niet gelijk aan de oplossingsverzameling van het volgende:
\[
\left\{
\begin{array}{c}
x+2y+z=0\\
y+z=0\\
x+y=0
\end{array}
\right.
\]
De oplossingsverzameling van bovenstaand stelsel heeft namelijk een dimensie meer.

\item
\begin{proof}
We tonen aan dat het stelsel bekomen kan worden uit het origineel stelsel door elementaire rijoperaties.
\[
\left\{
\begin{array}{c}
R_1\\R_2\\R_3
\end{array}
\right.
\overset{\begin{array}{c}
R_1 \leftrightarrow R_3\\
R_1 \leftrightarrow R_2
\end{array}}{\longrightarrow}
\left\{
\begin{array}{c}
R_2\\R_3\\R_1
\end{array}
\right.
\overset{\begin{array}{c}
R_3 \mapsto R_3+R_1\\
R_2 \mapsto R_2+R_1\\
R_1 \mapsto R_1+R_2+R_3\\
\end{array}}{\longrightarrow}
\left\{
\begin{array}{c}
R_1+3R_2+R_3\\R_2+R_3\\R_1+R_2
\end{array}
\right.
\]
\end{proof}

\end{enumerate}

\subsubsection*{(b)}
Te Bewijzen:
\[
\begin{vmatrix}
1 & a & a^2 & bcd\\
1 & b & b^2 & acd\\
1 & c & c^2 & abd\\
1 & d & d^2 & abc
\end{vmatrix}
=
-
\begin{vmatrix}
1 & a & a^2 & a^3\\
1 & b & b^2 & b^3\\
1 & c & c^2 & c^3\\
1 & d & d^2 & d^3
\end{vmatrix}
\]
\begin{proof}
    We zouden dit kunnen bewijzen door beide leden uit te werken, maar dit is natuurlijk zeer omslachtig. We zullen enkele slimme matrixbewerkingen gebruiken. Allereerst: het vermenigvuldigen van een rij of een kolom met een constante, vermenigvuldigt de waarde van de matrix met deze constante. We vinden:
\[
\begin{vmatrix}
1 & a & a^2 & bcd\\
1 & b & b^2 & acd\\
1 & c & c^2 & abd\\
1 & d & d^2 & abc
\end{vmatrix}
=
abcd \times
\begin{vmatrix}
    1 & a & a^2 & \frac{1}{a}\\
    1 & b & b^2 & \frac{1}{b}\\
    1 & c & c^2 & \frac{1}{c}\\
    1 & d & d^2 & \frac{1}{d}
\end{vmatrix}
=
\frac{abcd}{abcd} \times
\begin{vmatrix}
    a & a^2 & a^3 & 1 \\
    b & b^2 & b^3 & 1 \\
    c & c^2 & c^3 & 1 \\
    d & d^2 & d^3 & 1 
\end{vmatrix}
\]

We weten ook dat het omwisselen van kolommen in een determinant het teken doet omkeren. We moeten de laatste kolom nog drie keer wisselen met de rij die ervoor staat om ons uiteindelijke resultaat te bekomen.
\[
\begin{vmatrix}
    a & a^2 & a^3 & 1 \\
    b & b^2 & b^3 & 1 \\
    c & c^2 & c^3 & 1 \\
    d & d^2 & d^3 & 1 
\end{vmatrix}
=
(-1)^3 \times
\begin{vmatrix}
1 & a & a^2 & a^3\\
1 & b & b^2 & b^3\\
1 & c & c^2 & c^3\\
1 & d & d^2 & d^3
\end{vmatrix}
\]
\end{proof}


\end{document}
