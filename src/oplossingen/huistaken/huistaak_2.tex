\documentclass[lineaire_algebra_oplossingen.tex]{subfiles}
\begin{document}

\section{Huistaak 2}

\subsection{3.6.11}
\subsubsection*{Opgave}
Vind $k\in R$ zodat het volgende geldt.
\[
\begin{pmatrix}
1\\k\\5
\end{pmatrix}
\in 
vct
\left\lbrace
\begin{pmatrix}
1\\-3\\2
\end{pmatrix}
,
\begin{pmatrix}
2\\-1\\1
\end{pmatrix}
\right\rbrace
\]
\subsubsection*{Oplossing}
De $k$ die we zoeken, voldoet aan de volgende vergelijkingen.
\[
\left\lbrace
\begin{array}{c c c}
1 &= x   &+ 2y\\
5 &= 2x  &+ y\\
k &= -3x &- y\\
\end{array}
\right.
\]
Dit is een stelsel van drie vergelijkingen met drie onbekenden waarvan de oplossingsverzameling $V$ is.
\[
V = \{ (3,-1,-8) \}
\]
De gezochte $k$ is dus de volgende.
\[
k = -8
\]
\pagebreak
\subsection{3.6.15}
\subsubsection*{Opgave}
Zij $(\mathbb{R},\mathbb{V},+)$ een vectorruimte.
Zij $v_1,v_2,...,v_n \in \mathbb{V}$ en $v \in \mathbb{V}$ zodat $v_1,v_2,...,v_n$ lineair onafhankelijk zijn.

Toon aan:
$v_1,v_2,...,v_n, v$ zijn lineair afhankelijk $\Leftrightarrow$ $v$ is een lineaire combinatie van $v_1,v_2,...,v_n$

\subsubsection*{Te bewijzen}
Met behulp van de definitie van lineaire afhankelijkheid en lineaire combinatie schrijven we dit formeler op als volgt.
\[
\exists r_1,r_2,...,r_n,r:\ v_1\cdot r_1 + v_2\cdot r_2...v_n \cdot r_n + v \cdot r = 0
\]
\[\Leftrightarrow \]
\[\exists r_1,r_2,...,r_n,r:\ v_1\cdot r_1 + v_2\cdot r_2...v_n \cdot r_n = v \cdot r \]
\subsubsection*{Bewijs}
\begin{proof}
Rechtstreeks bewijs
\[
\exists r_1,r_2,...,r_n,r:\ v_1\cdot r_1 + v_2\cdot r_2...v_n \cdot r_n + v \cdot r = 0
\]
In woorden betekent dit dat elk van de $v$'s voor het gelijkheidsteken kan geschreven worden als een lineaire combinatie van de andere. $v$ dus ook.\\
Formeler:
\[
v_1\cdot r_1 + v_2\cdot r_2...v_n \cdot r_n + v \cdot r = 0
\]
\[
\Leftrightarrow v_1\cdot r_1 + v_2\cdot r_2...v_n \cdot r_n = - v \cdot r 
\]
\[\Leftrightarrow \exists r_1,r_2,...,r_n,r:\ v_1\cdot r_1 + v_2\cdot r_2...v_n \cdot r_n = v \cdot r \]
De laatste bewering is waar, want neem voor $r_1,r_2,...,r_n$ dezelfde als in de stap ervoor, en voor $r$ het tegenstelde van de $r$ uit de stap ervoor.
We gaan via equivalenties rechtstreeks van het linker lid van het te bewijzen naar het rechterlid. Hiermee is de stelling bewezen omdat $(\Leftrightarrow)$ transitief is.
\end{proof}


\end{document}