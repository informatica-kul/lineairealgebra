\documentclass[lineaire_algebra_oplossingen.tex]{subfiles}
\begin{document}

\section{Examen Januari 2011}

\subsection{Vraag 1 (Theorie)}
\subsubsection*{(a)}
Dit staat letterlijk in de cursus.
Zie Stelling 3.42 p. 109 (\ref{3.42}).

\subsubsection*{(b)}
Dit staat ook letterlijk in de cursus.
Zie Stelling 4.43 p. 164 (\ref{4.43}).

\subsection{Vraag 2 (Theorie)}
Dit staat letterlijk in de cursus.
Zie Stelling 6.36 p 243 (\ref{6.36}).


\subsection{Vraag 3}
\subsubsection*{(a)}
\begin{proof}
\[
\begin{vmatrix}
-\lambda & 0 & \cdots & 0 & -a_0\\
1 & -\lambda & \cdots & 0 & -a_1\\
0 & 1 & \cdots & 0 & -a_2\\
\vdots & \vdots & \ddots & \vdots & \vdots\\
0 & 0 & \cdots & 1 & -a_{n-1}-\lambda
\end{vmatrix}
=
-\lambda
\begin{vmatrix}
-\lambda & \cdots & 0 & -a_1\\
1 & \cdots & 0 & -a_2\\
\vdots & \ddots & \vdots & \vdots\\
0 & \cdots & 1 & -a_{n-1}-\lambda
\end{vmatrix}
-
\begin{vmatrix}
0 & \cdots & 0 & -a_0\\
1 & \cdots & 0 & -a_2\\
\vdots & \ddots & \vdots & \vdots\\
0 & \cdots & 1 & -a_{n-1}-\lambda
\end{vmatrix}
\]
Wanneer we dit uitwerken krijgen we het volgende.
\[
a_0 + \lambda(a_1 + \lambda(a_2 + \lambda(... + \lambda(a_{n-1} + \lambda)))\cdots)
\]
\end{proof}

\subsubsection*{(b)}
Zij $\lambda$ een willekeurige eigenwaarde van $L$. We weten bovendien dat $\lambda = 0$ geen nulpunt is van vorige functie.
\[
\begin{pmatrix}
-\lambda & 0 & \cdots & 0 & -a_0\\
1 & -\lambda & \cdots & 0 & -a_1\\
0 & 1 & \cdots & 0 & -a_2\\
\vdots & \vdots & \ddots & \vdots & \vdots\\
0 & 0 & \cdots & 1 & -a_{n-1}-\lambda
\end{pmatrix}
\]
\[
\rightarrow
\begin{pmatrix}
-\lambda & 0 & \cdots & 0 & -a_0\\
0 & -\lambda^2 & \cdots & 0 & -\lambda a_1-a_0\\
0 & 0 & \cdots & 0 & -\lambda a_2-\lambda(\lambda a_1-a_0)\\
\vdots & \vdots & \ddots & \vdots & \vdots\\
0 & 0 & \cdots & 0 & -a_0 + \lambda(a_1 + \lambda(a_2 + \lambda(... + \lambda(a_{n-1} + \lambda)))\cdots)
\end{pmatrix}
\]
Aangezien het element, het meest rechtsonder in die matrix een nul is, en er voor de rest geen nulrijen zijn, is $E_\lambda$ \'e'endimensionaal.

\subsubsection*{(c)}
\begin{proof}
Wanneer $L$ $n$ (\'e\'endimensionale) eigenruimten heeft, heeft $L$ $n$ verschillende eigenwaarden\footnote{Zie Stelling 5.18 p. 190 (\ref{5.18})}. Omdat er evenveel verschillende eigenwaarden zijn van $L$ als er dimensies zijn in $\mathbb{R}^n$ is het spectrum van $L$ enkelvoudig. Elke lineaire transformatie met een enkelvoudig spectrum is diagonaliseerbaar \footnote{Zie Gevolg 5.20 p. 191 (\ref{5.20})}.
\end{proof}

\subsection{Vraag 4}
\subsubsection*{(a)}
Fout. Tegenvoorbeeld:
Kies $U_1$ als de $x$-as. Kies $U_2$ als de $y$-as en kies $U_3$ als de identieke $\leftrightarrow y=x$.
\[
V = U_1 \oplus U_2
\]
Maar
\[
(U_3 \cap U_1) = \{\vec{0}\}
\]
\[
(U_3 \cap U_2) = \{\vec{0}\}
\]
\[
U_2 \neq \{\vec{0}\} \oplus \{\vec{0}\}
\]

\subsubsection*{(b)}
Juist.
\begin{proof}
We bewijzen dat elke vector in het linkerlid in de verzameling in het rechterlid zit, en omgekeerd.
\[
W_1 = U_1^\bot \cap U_2^\bot = \{\ v\ |\ \langle v,u_1\rangle = 0 \langle v,u_2\rangle = 0\ \}
\]
\[
W_2 = (U_1+U_2)^\bot = \{\ v\ |\ \langle v, u_1+u_2\rangle = 0\ \}
\]
\begin{itemize}
\item $\subseteq$\\
Kies een willekeurige $w_1 \in W_1$, een $u_1 \in U_1$ en een $u_2 \in U_2$. $\langle w_1,u_1\rangle = 0$ en $\langle w_1,u_2\rangle = 0$ geldt. Tellen we nu deze vergelijkingen bij elkaar op, dan krijgen we de volgende:
\[
\langle w_1,u_1\rangle + \langle w_1,u_2\rangle = 0
\]
We gebruiken nu de lineariteit van het inproduct in de tweede component.
\[
\langle w_1,u_1+u_2\rangle = 0
\]
Dit betekent dat $w_1 \in W_2$ geldt.

\item $\supseteq$\\
Kies een willekeurige $w_2 \in W_2$, een $u_1 \in U_1$ en een $u_2 \in U_2$. $\langle w_2,u_1+u_2\rangle = 0$ en geldt.
Omdat dit voor elke $u_1$ en $u_2$ geldt het ook als $u_1=0$ en als $u_2=0$. $w_2$ staat dus ook loodrecht op $u_1$ en $u_2$ apart en zit daarom in $W_1$.
\end{itemize}
\end{proof}

\subsection{Vraag 5}
\subsubsection*{(a)}
De matrix van $A$ ten opzichte van de standaardbasis noemen we $A_{\epsilon_3}$.
\[
A_{\epsilon_3}=
\begin{pmatrix}
1 & 1 & 0\\
1 & 1 & 0\\
1 & 0 & 1
\end{pmatrix}
\rightarrow
\begin{pmatrix}
1 & 0 & 1\\
0 & 1 & -1\\
0 & 0 & 0
\end{pmatrix}
\]
De kern van $A$ is dus de volgende verzameling:
\[
Ker(A) = 
\left\{
\begin{pmatrix}
-\lambda\\\lambda\\\lambda
\end{pmatrix}
\ |\ \lambda\in\mathbb{R}
\right\}
\]
Mits enig inzicht zien we ook dat het beeld van $A$ er als volgt uit ziet.
\[
Im(A) = 
\left\{
\begin{pmatrix}
\lambda\\\lambda\\\mu
\end{pmatrix}
\ |\ \lambda,\mu\in\mathbb{R}
\right\}
\]
De dimensie van $Ker(A)$ is $1$ en de dimensie van $Im(A)$ is $2$.
Kies de volgende verzamelingen respectievelijk als basis voor $Ker(A)$ en voor $Im(A)$.
\[
\left\{
\begin{pmatrix}
-1\\1\\1
\end{pmatrix}
\right\}
\]
\[
\left\{
\begin{pmatrix}
1\\1\\0
\end{pmatrix}
,
\begin{pmatrix}
0\\0\\1
\end{pmatrix}
\right\}
\]

\subsubsection*{(b)}
Kies $v_1 = (1,0,0)$ en $v_2 = (0,1,0)$ buiten de kern van $A$. De beelden van deze vectoren zijn $w_1=(1,1,1)$, $w_2=(1,1,0)$. Kies nu nog een lineair onafhankelijke $v_3=(-1,1,1)$ uit de kern van $A$ en nog een lineair onafhankelijke $w_3 = (1,0,0)$ \footnote{Voor meer details over deze procedure, zie \ref{specifieke_basissen_voor_lineaire_afbeelding}.}.
$V$ en $W$ zien er nu als volgt uit.
\[
V = 
\left\{
\begin{pmatrix}
1\\0\\0
\end{pmatrix}
,
\begin{pmatrix}
0\\1\\0
\end{pmatrix}
,
\begin{pmatrix}
-1\\1\\1
\end{pmatrix}
\right\}
\]
\[
W = 
\left\{
\begin{pmatrix}
1\\1\\1
\end{pmatrix}
,
\begin{pmatrix}
1\\1\\0
\end{pmatrix}
,
\begin{pmatrix}
1\\0\\0
\end{pmatrix}
\right\}
\]

\subsubsection*{(c)}
Nee, het beeld van $M_{V,W}$ heeft maar \'e\'en dimensie, en dat van $A$ heeft er twee.

\subsubsection*{(d)}
We zoeken de eigenwaarden van $A$.
\[
\begin{vmatrix}
1-\lambda & 1 & 0\\
1 & 1-\lambda & 0\\
1 & 0 & 1-\lambda
\end{vmatrix}
=
(1-\lambda)
\begin{vmatrix}
1-\lambda & 1\\
1 & 1-\lambda
\end{vmatrix}
=
(1-\lambda)((1-\lambda)^2-1)
\]
\[
(1-\lambda)(\lambda^2-2\lambda)
=
\lambda(1-\lambda)(\lambda-2)
\]
Omdat $A$ drie verschillende eigenwaarden heeft, heeft $A$ een enkelvoudig spectrum en is $A$ dus diagonaliseerbaar\footnote{Zie Gevolg 5.20 p. 191 (\ref{5.20})}. $A$ is dus gelijkvormig met een diagonaalmatrix en deze gelijkvormigheid wordt gerealiseerd door een basisverandering naar een basis van eigenvectoren\footnote{Zie Definitie 5.6 p. 181 (\ref{5.6})}.


\subsection{Vraag 6}
\subsubsection*{(a)}
$U$ is een strikte deelverzameling als de gegeven vectoren lineair afhankelijk zijn.
\[
\begin{vmatrix}
a & b & c\\
a & 2b & 3c\\
a & c & c
\end{vmatrix}
=
\begin{vmatrix}
a & b & c\\
0 & b & 2c\\
0 & c-b & 0
\end{vmatrix}
=
a
\begin{vmatrix}
b & 2c\\
c-b & 0
\end{vmatrix}
=
2ac(c-b)
\]
Deze determinant is nul wanneer $a=0\vee c=0\vee c=b$ geldt.

\subsubsection*{(b)}
\[
\left\{
\begin{pmatrix}
a\\b\\c\\
\end{pmatrix}
,
\begin{pmatrix}
a\\2b\\c\\
\end{pmatrix}
,
\begin{pmatrix}
a\\c\\c\\
\end{pmatrix}
\right\}
=
\left\{
\begin{pmatrix}
a\\b\\b\\
\end{pmatrix}
,
\begin{pmatrix}
a\\2b\\b\\
\end{pmatrix}
,
\begin{pmatrix}
a\\b\\b\\
\end{pmatrix}
\right\}
\]
We kunnen deze verzameling uitdunnen tot een basis voor $U$\footnote{Zie Stelling 3.37 p. 107 (\ref{3.37})}. Schrap de laatste, omdat ze gelijk is aan de eerste, zodat de volgende basis voor $U$ overblijft.
\[
\left\{
\begin{pmatrix}
a\\b\\b\\
\end{pmatrix}
,
\begin{pmatrix}
a\\2b\\b\\
\end{pmatrix}
\right\}
\]

\subsubsection*{(c)}
We zoeken de vectoren $v= (x,y,z)$ zodat $v$ zowel loodrecht op $(a,b,b)$ als op $(a,2b,b)$ staat.
\[
\begin{pmatrix}
a & b & b\\
a & 2b & b
\end{pmatrix}
X=\vec{0}
\]
\[
\rightarrow
\begin{pmatrix}[ccc|c]
a & 0 & b & 0\\
0 & b & 0 & 0
\end{pmatrix}
\]
\[
U^\bot = 
\left\{
\begin{pmatrix}
-\lambda\frac{b}{a} \\ 0 \\ \lambda
\end{pmatrix}
\ |\ \lambda\in\mathbb{R}
\right\}
\]


\end{document}