\documentclass[lineaire_algebra_oplossingen.tex]{subfiles}
\begin{document}

\section{Examen Januari 2010}
\subsection{Vraag 1 (Theorie)}
\subsubsection*{(a)}
Dit bewijs staat letterlijk in de cursus. Zie Stelling 3.37 p. 107 (\ref{3.37}).

\subsubsection*{(b)}
Afgezien van het feit dat een algoritme een probleem zou hebben met oneindigdimensionale vectorruimten, zou dit geen probleem mogen vormen.

\subsection{Vraag 2 (Theorie)}
Dit bewijs staat letterlijk in de cursus. Zie Stelling 5.18 p. 190 (\ref{5.18}).

\subsection{Vraag 3}
\subsubsection*{(a)}
We lossen volgend stelsel op.
\[
\begin{pmatrix}[cccccccccccccccc|c]
1 & 1 & 0 & 0 & 0 & 0 & 0 & 0 & 0 & 0 & 0 & 0 & 0 & 0 & 0 & 0 & 0\\
0 & 0 & 0 & 0 & 1 & 1 & 0 & 0 & 0 & 0 & 0 & 0 & 0 & 0 & 0 & 0 & 1\\
0 & 0 & 0 & 0 & 0 & 0 & 0 & 0 & 1 & 1 & 0 & 0 & 0 & 0 & 0 & 0 & 0\\
0 & 0 & 0 & 0 & 0 & 0 & 0 & 0 & 0 & 0 & 0 & 0 & 1 & 1 & 0 & 0 & -1\\
1 & 0 & 1 & 0 & 0 & 0 & 0 & 0 & 0 & 0 & 0 & 0 & 0 & 0 & 0 & 0 & 1\\
0 & 0 & 0 & 0 & 1 & 0 & 1 & 0 & 0 & 0 & 0 & 0 & 0 & 0 & 0 & 0 & 1\\
0 & 0 & 0 & 0 & 0 & 0 & 0 & 0 & 1 & 0 & 1 & 0 & 0 & 0 & 0 & 0 & 1\\
0 & 0 & 0 & 0 & 0 & 0 & 0 & 0 & 0 & 0 & 0 & 0 & 1 & 0 & 1 & 0 & 0\\
0 & 1 & 0 & -1 & 0 & 0 & 0 & 0 & 0 & 0 & 0 & 0 & 0 & 0 & 0 & 0 & 0\\
0 & 0 & 0 & 0 & 0 & 1 & 0 & -1 & 0 & 0 & 0 & 0 & 0 & 0 & 0 & 0 & 0\\
0 & 0 & 0 & 0 & 0 & 0 & 0 & 0 & 0 & 1 & 0 & -1 & 0 & 0 & 0 & 0 & 0\\
0 & 0 & 0 & 0 & 0 & 0 & 0 & 0 & 0 & 0 & 0 & 0 & 0 & 1 & 0 & -1 & 0\\
1 & 1 & 1 & 0 & 0 & 0 & 0 & 0 & 0 & 0 & 0 & 0 & 0 & 0 & 0 & 0 & 0\\
0 & 0 & 0 & 0 & 1 & 1 & 1 & 0 & 0 & 0 & 0 & 0 & 0 & 0 & 0 & 0 & 0\\
0 & 0 & 0 & 0 & 0 & 0 & 0 & 0 & 1 & 1 & 1 & 0 & 0 & 0 & 0 & 0 & 0\\
0 & 0 & 0 & 0 & 0 & 0 & 0 & 0 & 0 & 0 & 0 & 0 & 1 & 1 & 1 & 0 & 0\\
\end{pmatrix}
\]
De matrix van $A$ is dus de volgende:
\[
A =
\begin{pmatrix}
1 & -1 & 0 & -1\\
2 & -1 & -1 & -1\\
1 & -1 & 0 & -1\\
-1 & 0 & 1 & 0
\end{pmatrix}
\]

\subsubsection*{(b)}
Ja.
%TODO geef er.

%\subsection{Vraag 4}
%\subsubsection*{(a)}
%TODO
%\subsubsection*{(b)}
%TODO

\subsection{Vraag 5}
\[
\begin{vmatrix}
10+c-\lambda & c-2 & 4-2c\\
c-2 & 10+c-\lambda & 4-2c\\
4-2c & 4-2c & 4+4c-\lambda\\
\end{vmatrix}
=
0
\]

\subsection{Vraag 6}
Dit is een oefening die letterlijk in de cursus staat. Zie oefening 18 van hoofdstuk 6 (\ref{oef:6.18}).


\subsection{Vraag 7}
\begin{itemize}
\item $\Rightarrow$\\
\begin{itemize}
\item Stel dat $f$ surjectief is.\\
\[
rang(f) = dim(W)
\]
De rang van $g$ is hoogstens de dimensie van $W$. De rang van $g$ is dus ook hoogstens de rang van $f$. 
\item Stel dat $f$ injectief is.\\
\[
rang(f) = dim(V)
\]
De rang van $g$ is hoogstens de dimensie van $V$. De rang van $g$ is dus ook hoogstens de rang van $f$. 
\end{itemize}
\item $\Leftarrow$\\
Stel $rang(g) \le rang(f)$ maar $f$ is noch injectief, noch surjectief.
Nu gelden volgende beweringen.
\[
rang(f) \le dim(V) \text{ en } rang(f) \le dim(W)
\]
\begin{itemize}
\item Stel $dim V \le dim W$\\
Nu bestaat er een surjectieve $g$ zodat $rang(g) = dim(V) \le rang (g)$.
\item Stel $dim V \ge dim W$\\
Nu bestaat er een injectieve $g$ zodat $rang(g) = dim(W) \le rang(f)$.
\end{itemize}
\end{itemize}

\end{document}
