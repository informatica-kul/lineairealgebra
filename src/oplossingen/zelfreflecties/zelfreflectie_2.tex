\documentclass[lineaire_algebra_oplossingen.tex]{subfiles}
\begin{document}

\section{Zelfreflectie 2}
\subsection{Oefening 1}
\[
C_{11} = 
\begin{vmatrix}
5 & 6 \\
8 & 9
\end{vmatrix}
=-3
\]
\[
C_{22}= 
\begin{vmatrix}
1 & 3 \\
7 & 9
\end{vmatrix}
=
18
\]
\[
C_{12} =
\begin{vmatrix}
4 & 6 \\
7 & 9
\end{vmatrix}
=-6
\]
\[
M_{11} = 
\begin{pmatrix}
5 & 6 \\
8 & 9
\end{pmatrix}
\]
\[
M_{22}= 
\begin{pmatrix}
1 & 3 \\
7 & 9
\end{pmatrix}
\]
\[
M_{12} =
\begin{pmatrix}
4 & 6 \\
7 & 9
\end{pmatrix}
\]
\subsection{Oefening 2}
\subsubsection*{Te bewijzen}
$A$ is inverteerbaar $\Rightarrow$ $f(A^{-1}) = f(A)^{-1}$
\subsubsection*{Bewijs}
``$A$ is inverteerbaar'' betekent het volgende.
\[
\exists B: A\cdot B = I_n = B\cdot A
\]
Noteer $B = A^{-1}$ 
\begin{proof}
Rechtstreeks bewijs.\\
Noteer de inverse van $A$ als $B$.
\[
f(B) = f(A)^{-1}
\]
\[
f(B)\cdot f(A) = f(A)^{-1}f(A)
\]
Definitie determinant afbeelding
\[
f(B\cdot A) = 1
\]
$A$ is inverteerbaar met $B$ als inverse.
\[
f(I_n) = 1
\]
\[
1 = 1
\]
\[
True
\]
\end{proof}

\subsection{Oefening 3}
\subsubsection*{Te bewijzen}
\[
\begin{vmatrix}
1 & x_1 & x_1^2 & \hdots & x_1^{n-1}\\
1 & x_2 & x_2^2 & \hdots & x_2^{n-1}\\
\vdots &\vdots &\vdots & \ddots & \vdots \\
1 & x_n & x_n^2 & \hdots & x_n^{n-1}\\
\end{vmatrix}
=
\prod_{i>j}(x_i-x_j)
\]
\subsubsection*{Bewijs}
\begin{proof}
Bewijs door inductie.\\
\emph{stap 1: (basis)}
De beweging geldt voor $n=2$.
\[
\begin{vmatrix}
1 & x_1\\
1 & x_2
\end{vmatrix}
=
(x_2-x_1) = \prod_{i>j}(x_i-x_j)
\]
\emph{stap 2: (inductiestap)}
Stel dat de bewering geldt voor een bepaalde $k$. We bewijzen nu dat de bewering geldt voor $k+1$.
De determinant van $V_{k+1}$ ziet er dan als volgt uit
\[
\det(V_{k+1}) =
\begin{vmatrix}
1 & x_1 & x_1^2 & \hdots & x_1^{k}\\
1 & x_2 & x_2^2 & \hdots & x_2^{k}\\
\vdots &\vdots &\vdots & \ddots & \vdots \\
1 & x_{k+1} & x_{k+1}^2 & \hdots & x_{k+1}^{k}\\
\end{vmatrix}
\]
We trekken nu van elke rij behalve de eerste de eerste af. De waarde van de determinant verandert niet, en we krijgen de volgende uitdrukking.
\[
\det(V_{k+1}) =
\begin{vmatrix}
1 & x_1 & x_1^2 & \hdots & x_1^{k}\\
0 & x_2-x_1 & x_2^2-x_1^2 & \hdots & x_2^{k}-x_1^{k}\\
\vdots &\vdots &\vdots & \ddots & \vdots \\
0 & x_{k+1}-x_1 & x_{k+1}^2-x_1^2  & \hdots & x_{k+1}^{k}-x_1^{k}\\
\end{vmatrix}
\]
Nu trekken we(in volgorde, van achter naar voor) $x_1$ keer kolom $n-1$ van kolom $n$ af voor elke $n > 1$. We krijgen dan de volgende uitdrukking, na enig gefoefel.
\[
\det(V_{k+1}) =
\begin{vmatrix}
1 & 0 & 0 & \hdots & 0\\
0 & x_2-x_1 & (x_2-x_1)x_2 & \hdots & (x_2-x_1)x_2^{k-1}\\
\vdots &\vdots &\vdots & \ddots & \vdots \\
0 & x_{k+1}-x_1 & (x_{k+1}-x_1)x_{k+1}^2  & \hdots & (x_{k+1}-x_1)x_{k+1}^{k-1}\\
\end{vmatrix}
\]
In elke rij $j \neq 1$ kunnen we $x_j-x_1$ afzonderen.
\[
\det(V_{k+1}) =
\prod_{k=2}^n(x_k-x_1)
\begin{vmatrix}
1 & 0 & 0 & \hdots & 0\\
0 & 1 & x_2 & \hdots & x_2^{k-1}\\
\vdots &\vdots &\vdots & \ddots & \vdots \\
0 & 1 & x_{k+1}^2  & \hdots & x_{k+1}^{k-1}\\
\end{vmatrix}
\]
De determinant die na het product nog overblijf is een determinant van vandermonde $V_k$.
\[
\det(V_{k+1}) =
\prod_{k=2}^n(x_k-x_1)
\prod_{i>j,j>2}(x_i-x_j)
= \prod_{i>j}(x_i-x_j)
\]
\end{proof}

\subsection{Oefening 4}
Als $a=0$ kunnen we de beschreven determinant ofwel omvormen, tot een matrix waarbij de nieuwe $a \neq 0$ ofwel zal de determinant nul worden omdat er dan een hele kolom/rij nullen in de matrix staat. 

\subsection{Oefening 5}
\[
\det\left(
\begin{matrix}
a_{11} & a_{12}\\
a_{21} & a_{22}
\end{matrix}
\right)
\]
\[
\det\left(
\begin{matrix}
a_{11} & a_{12} & a_{13}\\
a_{21} & a_{22} & a_{23}\\
a_{31} & a_{32} & a_{33}
\end{matrix}
\right)
\]
Bewijs elke eigenschap die nodig is voor de regel van sarrus, gebruik de regel van sarrus. Deze oefening is helemaal niet nuttig.

\subsection{Oefening 6}
Ja, we weten dat $A$ inverteerbaar is, want $A$ komt uit het stelsel van Cramer. Zie nu p 72 gevolg 2.22.
\[
X = \frac{1}{\det(A)}\text{adj}(A)\cdot B = A^{-1}\cdot B
\]
\[
A\cdot X = B
\]

\subsection{Oefening 7}
\subsubsection{2.3.2}
$f(E_1) = 1$.\\
$f(I) = 1$. Als we de elemetaire operatie uitvoeren op $I$ krijgen we $E\cdot I$. Dus $f(I) = f(E\cdot I) = f(E) f(I) = f(E)=1$.\\
$f(E_2)=-1$.\\
$E_2$ verkrijgen we door in $I$ twee rijen om te wisselen. $f(I) = 1$ dus $f(E_2)=-1$ want als we twee rijen verwisselen verandert de determinant van teken (definitie determinant afbeelding).
$f(E_3) = \lambda$\\
$E_3$ verkrijgen we door in $I$ een rij $r$ te vervangen door $\lambda r$. Als we stelling 2.3.1 $\lambda-1$ keer toepassen krijgen we $f(E_3) = \lambda f(I) = \lambda$.
\subsubsection{2.3.3}
Dit is een bijzonder lang bewijs voor wat het is. Zie 
\begin{center}
\url{http://www.proofwiki.org/wiki/Determinant_of_Matrix_Product}
\end{center}

\end{document}